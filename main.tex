\documentclass[12pt,a4paper]{report}

% --- Pakiety ---
\usepackage[polish]{babel}
\usepackage[T1]{fontenc}
\usepackage[utf8]{inputenc}
\usepackage{lmodern}
\usepackage{geometry}
\usepackage{graphicx}
\usepackage{float}
\usepackage{booktabs}
\usepackage{hyperref}
\usepackage{fancyhdr}
\usepackage{setspace}
\usepackage{csquotes}
\usepackage{biblatex}

\geometry{margin=2.5cm}
\addbibresource{bibliografia.bib}

% --- Wczytanie danych autora (POZA repo) ---
\input{autor.tex}

% --- Nagłówki ---
\pagestyle{fancy}
\fancyhf{}
\fancyhead[L]{\leftmark}
\fancyhead[R]{Autor: \Autor}
\fancyfoot[C]{\thepage}

% --- Dane dokumentu ---
\title{Długie dokumenty\\
\large Analiza formy, struktury i procesu}
\author{\Autor}
\date{Styczeń 2026}

\begin{document}

\begin{titlepage}
    \centering
    \vspace*{3cm}
    {\Huge\bfseries Długie dokumenty\par}
    \vspace{1cm}
    {\Large Analiza formy, struktury i procesu\par}
    \vspace{2cm}
    {\Large Autor: \Autor\par}
    \vfill
    {\large Data powstania dokumentu: \today\par}
\end{titlepage}

% --- Streszczenie ---
\chapter*{Streszczenie}
\addcontentsline{toc}{chapter}{Streszczenie}

Dokument analizuje proces tworzenia i wypełniania długich dokumentów
oraz znaczenie struktury, cierpliwości i konsekwencji w pracy z tekstem.
Przedstawiono przykłady tabel, ilustracji oraz list, które porządkują treść
i ułatwiają odbiór informacji. Wszystkie źródła w bibliografii pochodzą
z publicznie dostępnych materiałów online.

\tableofcontents
\newpage

% --- Rozdział 1 ---
\chapter{Wprowadzenie do długich dokumentów}

Tworzenie i wypełnianie długich dokumentów wymaga planowania
i systematyczności \cite{dokumenty}. Ich odbiór i analiza przez czytelnika
może być utrudniona bez konsekwentnej struktury.

\section{Podział dokumentów}

Długie dokumenty można podzielić na:
\begin{itemize}
    \item dokumenty naukowe i raporty,
    \item eseje i prace akademickie,
    \item powieści i narracje rozbudowane.
\end{itemize}

\section{Znaczenie struktury}

Nagłówki, akapity, tabele i ilustracje pełnią funkcję punktów orientacyjnych
dla czytelnika. Ułatwiają zrozumienie treści i organizację informacji.

% --- Ilustracja ---
\begin{figure}[H]
    \centering
    \includegraphics[width=0.7\textwidth]{rysunek1.jpg}
    \caption{Przykładowa ilustracja w dokumencie}
    \label{fig:ilustracja}
\end{figure}

% --- Rozdział 2 ---
\chapter{Proces wypełniania dokumentów}

Wypełnianie dokumentów wymaga cierpliwości i dokładności \cite{dokumenty}.

\section{Etapy pracy}

\begin{enumerate}
    \item Analiza wymagań dokumentu,
    \item Zebranie danych i materiałów,
    \item Stopniowe uzupełnianie treści,
    \item Kontrola i korekta.
\end{enumerate}

\section{Tabele i listy}

Tabele i listy ułatwiają porządkowanie danych i porównywanie informacji.

\begin{table}[H]
\centering
\caption{Przykładowa tabela w dokumencie}
\begin{tabular}{lll}
\toprule
Sekcja & Typ danych & Dokładność \\
\midrule
A & Tekst & Wysoka \\
B & Liczby & Bardzo wysoka \\
C & Daty & Jednolity format \\
\bottomrule
\end{tabular}
\end{table}

\begin{itemize}
    \item Punkt listy nienumerowanej 1
    \item Punkt listy nienumerowanej 2
\end{itemize}

% --- Rozdział 3 ---
\chapter{Dokument jako forma refleksyjna}

Dokument może komentować własną strukturę i proces tworzenia.
Staje się wtedy nie tylko narzędziem, ale również zapisem metodologicznym
\cite{esej,dokumenty}.

% --- Wnioski ---
\chapter{Wnioski}

Do realizacji zadania wybrano klasę \texttt{report}, ponieważ:
\begin{itemize}
    \item umożliwia logiczny podział na rozdziały,
    \item jest odpowiednia dla dokumentów akademickich i analitycznych,
    \item pozwala na automatyczne tworzenie spisu treści i bibliografii.
\end{itemize}

Źródła dokumentu dostępne są w publicznym repozytorium Git:
\begin{center}
\url{https://github.com/Tajemniczy30000065/latex}
\end{center}


\printbibliography

\end{document}
